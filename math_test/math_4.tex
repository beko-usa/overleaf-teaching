%数学テストテンプレート  タイトルと問題を書き換えて使用する

\documentclass[a4paper,14pt]{article}  % これは絶対先頭に必要なやつ 用紙サイズ,本文フォント大きさ
\usepackage{setspace}                                %余白をとるためのパッケージ
\usepackage{xeCJK, xcolor, tcolorbox, amsmath, amssymb, geometry}
\geometry{top=18mm,bottom=20mm,left=20mm,right=20mm} %ページマージン


% ===== 色定義(RGB) =====
\definecolor{myred}{RGB}{230,0,18}  %ミャクミャクの赤
\definecolor{myblue}{RGB}{0,104,180}  %ミャクミャクの青

% ===== フォント設定 =====
% 日本語フォント設定(丸ゴ風)
\setCJKmainfont{Noto Sans CJK JP} %日本語の本文
\setCJKsansfont{Noto Sans CJK JP} %sans-serifに対応する日本語フォント
\setCJKmonofont{Noto Sans CJK JP} %等幅フォント(コード等)
%\setCJKmainfont{IPAMincho}  % 標準(明朝)

%英文フォント設定
\setmainfont{Noto Sans}

\sffamily % ← 全体をサンセリフ体(丸ゴ風)に

\setCJKfamilyfont{mincho}{IPAMincho} %フォントにファミリ名を付ける
  %プリアンブルは別ファイルから読み込み(共通設定)

\begin{document}

% ===== タイトルブロック =====
\begin{tcolorbox}[
  colback=myblue!5,     % 背景:薄い青
  colframe=myblue,      % 枠:本青
  boxrule=1pt,
  arc=3mm,
  width=\textwidth
]

\begin{center}                                               %センター揃え
{\Large  \textcolor{myblue}{数学テスト No.4}}\\[0.4em]         %テストのタイトル,ナンバー
{ \CJKfamily{mincho} 式の計算}\\[0.8em]               %テストの内容 ここだけ明朝
\textcolor{myred}{日付:\underline{\hspace{3cm}} \hfill 点数:\underline{\hspace{2cm}}}
\end{center}
\end{tcolorbox}

\vspace{1em}  %下方にスペース

% ===== 問題部分 =====

% iワーク中2 p32,33,40,41より20題抜粋

次の計算をしなさい.
\begin{enumerate}                                        %自動でナンバーがつく
 \begin{spacing}{5.2}                                      %問題部分の改行幅
  \item $(5a^2+7ab+1)-(a^2-3ab)$
  \item $\dfrac{1}{3}a-2b+7b+\dfrac{1}{6}a$
  \item $-8(-\dfrac{1}{2}a+\dfrac{3}{4}b)$
  \item $(6x-12y+6)\div6$
  \item $(2x^2-7xy)-(x^2-8xy)$
  \item $(10x-15y+5)\times\dfrac{2}{5}$
  \item $\dfrac{3}{2}(8x-2y)$
  \item $(4ab+6a)\div(-\dfrac{2}{5})$
  \item $-2(5m+n-7)$
  \item $\dfrac{2x-5}{6}+\dfrac{3x-y}{4}$
  \item $4x+y-\dfrac{x+2y}{3}$
  \item $(-8xy)\times(-5y^2)$
  \item $5(2a^2-3a)-4(3a^2+7a)$
  \item $(-9ab)\div\dfrac{3}{5}a^3b$
  \item $a^2\times2b\div ab$
  \item $6x\div2xy\times3xy^2$
  \item $(-3x)^2 \div(-2x)\times6x$
  \item $\dfrac{5a+3b}{8}-\dfrac{2a+7b}{4}$
  \item $\dfrac{1}{2}(4x+6y)+\dfrac{2}{3}(6x-9y)$  
  \item $(-3a)^3\times \dfrac{2}{3}a$
 \end{spacing} 
\end{enumerate}

\vspace{2em}

% ===== 解答欄デザイン(赤枠) =====
%\begin{tcolorbox}[
%  colback=white,
%  colframe=myred,
%  boxrule=0.8pt,
%  arc=2mm,
%  width=\textwidth
%]
%\vspace{6cm} % 解答スペースの高さ
%\end{tcolorbox}

\end{document}
