\documentclass[a4paper,14pt]{article}  % これは絶対先頭に必要なやつ 用紙サイズ,本文フォント大きさ
\usepackage{setspace}                                %余白をとるためのパッケージ
\usepackage{xeCJK, xcolor, tcolorbox, amsmath, amssymb, geometry}
\geometry{top=18mm,bottom=20mm,left=20mm,right=20mm} %ページマージン


% ===== 色定義(RGB) =====
\definecolor{myred}{RGB}{230,0,18}  %ミャクミャクの赤
\definecolor{myblue}{RGB}{0,104,180}  %ミャクミャクの青

% ===== フォント設定 =====
% 日本語フォント設定(丸ゴ風)
\setCJKmainfont{Noto Sans CJK JP} %日本語の本文
\setCJKsansfont{Noto Sans CJK JP} %sans-serifに対応する日本語フォント
\setCJKmonofont{Noto Sans CJK JP} %等幅フォント(コード等)
%\setCJKmainfont{IPAMincho}  % 標準(明朝)

%英文フォント設定
\setmainfont{Noto Sans}

\sffamily % ← 全体をサンセリフ体(丸ゴ風)に

\setCJKfamilyfont{mincho}{IPAMincho} %フォントにファミリ名を付ける
  %プリアンブルは別ファイルから読み込み(共通設定)

\begin{document}

% ===== タイトルブロック =====
\begin{tcolorbox}[
  colback=myblue!5,     % 背景:薄い青
  colframe=myblue,      % 枠:本青
  boxrule=1pt,
  arc=3mm,
  width=\textwidth
]


\begin{center}                                               %センター揃え
{\Large  \textcolor{myblue}{数学テスト No.3}}\\[0.4em]         %テストのタイトル,ナンバー
{ \CJKfamily{mincho} 文字式の利用}\\[0.8em]               %テストの内容 ここだけ明朝
\textcolor{myred}{日付:\underline{\hspace{3cm}} \hfill 点数:\underline{\hspace{2cm}}}
\end{center}
\end{tcolorbox}

\vspace{1em}  %下方にスペース

% ===== 問題部分 =====

% iワーク中2 p22~24より20題抜粋

\begin{enumerate}                                        %自動でナンバーがつく
 \begin{spacing}{3.5}                                      %問題部分の改行幅
  \item 次の等式を[ ]の中の文字について解きなさい.
    \begin{enumerate}  
      \item $m + n = 3 \quad [m]$
      \item $\dfrac{1}{2}xy = 10 \quad [y]$
      \item $V = \dfrac{1}{3}Sh \quad [h]$
      \item $3x + 5y - 8 = 0 \quad [x]$
      \item $3ab = 6 \quad [b]$
      \item $c = 4(a+b) \quad [a]$
      \item $\dfrac{2p+q}{5}=r \quad [q]$      
    \end{enumerate}
  \end{spacing}  
  \vspace{1em}    
  \begin{spacing}{1.8} 
  \item 3つの続いた整数の和は3の倍数になる.このわけを,文字を使って説明する.(\quad)は式を,【\quad】には日本語を入れなさい.
    \begin{tcolorbox}[colback=white, colframe=black, boxrule=0.4pt,                   arc=2mm, left=2mm, right=2mm, top=1mm, bottom=1mm]
    3つの続いた整数のうち,もっとも小さい整数をnとすると,3つの続いた整数は
    $n,(\rule{2cm}{0.1mm}),(\rule{2cm}{0.1mm})$ と表される.それらの和は,
      \begin{align*}
          n + (\rule{2cm}{0.1mm} ) + (\rule{2cm}{0.1mm}) &= (\rule{2cm}{0.1mm})\\
          &=3(\rule{2cm}{0.1mm})
      \end{align*}
    $n+1$は【\rule{2cm}{0.1mm}】だから,$3(\rule{2cm}{0.1mm})$は3の倍数である.
    したがって,3つの続いた整数の和は3の倍数になる.
    \end{tcolorbox}
 
  \item 1,3,5のような差が2である3つの整数の和は3の倍数になる.このわけを,文字を使って説明しなさい.
    \begin{tcolorbox}[colback=white, colframe=black, boxrule=0.4pt,                   arc=2mm, left=2mm, right=2mm, top=1mm, bottom=1mm]
      \vspace{9cm}
    \end{tcolorbox}
  \vspace{4em}
  \item 偶数と奇数の和は奇数になる.このわけを,文字を使って説明しなさい.
    \begin{tcolorbox}[colback=white, colframe=black, boxrule=0.4pt,                   arc=2mm, left=2mm, right=2mm, top=1mm, bottom=1mm]
      \vspace{9cm}
    \end{tcolorbox}
 \end{spacing} 
\end{enumerate}

\vspace{2em}

\end{document}