\documentclass[a4paper,14pt]{article}                %用紙サイズ,本文フォント大きさ
\usepackage{setspace}                                %余白をとるためのパッケージ
\usepackage{xeCJK, xcolor, tcolorbox, amsmath, amssymb, geometry}
\geometry{top=18mm,bottom=20mm,left=20mm,right=20mm} %ページマージン


% ===== 色定義(RGB) =====
\definecolor{myred}{RGB}{230,0,18}  %ミャクミャクの赤
\definecolor{myblue}{RGB}{0,104,180}  %ミャクミャクの青

% ===== フォント設定 =====
% 日本語フォント設定(丸ゴ風)
\setCJKmainfont{Noto Sans CJK JP} %日本語の本文
\setCJKsansfont{Noto Sans CJK JP} %sans-serifに対応する日本語フォント
\setCJKmonofont{Noto Sans CJK JP} %等幅フォント(コード等)
%\setCJKmainfont{IPAMincho}  % 標準(明朝)

%英文フォント設定
\setmainfont{Noto Sans}

\sffamily % ← 全体をサンセリフ体(丸ゴ風)に

\setCJKfamilyfont{mincho}{IPAMincho} %フォントに名前を付ける


\begin{document}

% ===== タイトルブロック =====
\begin{tcolorbox}[
  colback=myblue!5,     % 背景:薄い青
  colframe=myblue,      % 枠:本青
  boxrule=1pt,
  arc=3mm,
  width=\textwidth
]

\begin{center}                                               %センター揃え
{\Large  \textcolor{myblue}{数学テスト No.1}}\\[0.4em]         %テストのタイトル,ナンバー
{ \CJKfamily{mincho} 単項式の乗法と除法}\\[0.8em]               %テストの内容 ここだけ明朝
\textcolor{myred}{日付:\underline{\hspace{3cm}} \hfill 点数:\underline{\hspace{2cm}}}
\end{center}
\end{tcolorbox}

\vspace{1em}  %下方にスペース

% ===== 問題部分 =====

% iワーク中2 p16-18より20題抜粋

\begin{enumerate}                                        %自動でナンバーがつく
 \begin{spacing}{5}                                      %問題部分の改行幅
  \item $(-5y)^2 \times 2$
  \item $xy^2 \div \dfrac{1}{2}xy$
  \item $- \displaystyle \frac{5}{18}xy^2 \div \frac{10}{27}xy$
  \item $-4ab^4 \times 7a \div (-14b)$
  \item $(-24x^2y) \div 6x \div (-2y)$
  \item $(-a^4) \div (-a) \div \displaystyle \frac{a^2}{2}$
  \item $a^2b \div \displaystyle \frac{1}{3}ab $
  \item $-8x^2 \div 2x$
  \item $\displaystyle \frac{7}{12}a^4 \div \frac{21}{4}a$
  \item $(3x)^2$
  \item $\displaystyle \frac{1}{3}x \times \frac{2}{5}x$
  \item $3a \times (-2b)$
  \item $\displaystyle x^2y \times y \div \frac{1}{3}xy^2$
  \item $\displaystyle -12ab \div 3a \times \frac{1}{4}b$
  \item $a^4 \times a$
  \item $(-8x) \times (-2x)$
  \item $\displaystyle -\frac{2}{3}a^2b \div \frac{4}{9}a$
  \item $-4xy \div (-2x)$
  \item $\displaystyle \frac{1}{2}x \times (2x)^2$
  \item $(-x^2) \times 4x$
 \end{spacing} 
\end{enumerate}

\vspace{2em}

% ===== 解答欄デザイン(赤枠) =====
%\begin{tcolorbox}[
%  colback=white,
%  colframe=myred,
%  boxrule=0.8pt,
%  arc=2mm,
%  width=\textwidth
%]
%\vspace{6cm} % 解答スペースの高さ
%\end{tcolorbox}

\end{document}
