%数学テストテンプレート  タイトルと問題を書き換えて使用する
\documentclass[a4paper,14pt]{article}  % これは絶対先頭に必要なやつ 用紙サイズ,本文フォント大きさ
\usepackage{setspace}                                %余白をとるためのパッケージ
\usepackage{xeCJK, xcolor, tcolorbox, amsmath, amssymb, geometry}
\geometry{top=18mm,bottom=20mm,left=20mm,right=20mm} %ページマージン


% ===== 色定義(RGB) =====
\definecolor{myred}{RGB}{230,0,18}  %ミャクミャクの赤
\definecolor{myblue}{RGB}{0,104,180}  %ミャクミャクの青

% ===== フォント設定 =====
% 日本語フォント設定(丸ゴ風)
\setCJKmainfont{Noto Sans CJK JP} %日本語の本文
\setCJKsansfont{Noto Sans CJK JP} %sans-serifに対応する日本語フォント
\setCJKmonofont{Noto Sans CJK JP} %等幅フォント(コード等)
%\setCJKmainfont{IPAMincho}  % 標準(明朝)

%英文フォント設定
\setmainfont{Noto Sans}

\sffamily % ← 全体をサンセリフ体(丸ゴ風)に

\setCJKfamilyfont{mincho}{IPAMincho} %フォントにファミリ名を付ける
 %プリアンブルは別ファイルから読み込み

\begin{document}

% ===== タイトルブロック =====
\begin{tcolorbox}[
  colback=myblue!5,     % 背景:薄い青
  colframe=myblue,      % 枠:本青
  boxrule=1pt,
  arc=3mm,
  width=\textwidth
]

\begin{center}                                               %センター揃え
{\Large  \textcolor{myblue}{数学テスト No.2 こたえ}}\\[0.4em]         %テストのタイトル,ナンバー
{ \CJKfamily{mincho} 単項式の乗法と除法,式の値}\\[0.8em]               %テストの内容 ここだけ明朝
\textcolor{myred}{日付:\underline{\hspace{3cm}} \hfill 点数:\underline{\hspace{2cm}}}
\end{center}
\end{tcolorbox}

\vspace{1em}  %下方にスペース

% ===== 問題部分 =====

% iワーク中2 p19~21より20題抜粋
 次の計算をしなさい.
\begin{enumerate}                                        %自動でナンバーがつく
 \begin{spacing}{3.8}%問題部分の改行幅

  \item $ab \times 5a^2 = \textcolor{myred}{5a^3b}$
  \item $18xy \times \left(-\dfrac{1}{3}x \right) = \textcolor{myred}{-6x^2y}$
  \item $15a^2b \div (-3ab) = \textcolor{myred}{-5a}$
  \item $(3x)^2 \div \dfrac{1}{2}xy \times y = \textcolor{myred}{72x}$
  \item $3ab \div 2a \times (-4a^2b) = \textcolor{myred}{-6a^2b^2}$
  \item $-6a^2b \div \dfrac{2}{3}a \div (-b)^2 = \textcolor{myred}{-\dfrac{9a}{b}}$
  \item $a \times b^2 \div ab = \textcolor{myred}{b}$
  \item $(-5m)^2 = \textcolor{myred}{25m^2}$
  \item $8a \times (-a^2) = \textcolor{myred}{-8a^3}$
  \item $\dfrac{3}{4}b^2c \div \dfrac{5}{8}bc^2 = \textcolor{myred}{\dfrac{6b}{5c}}$
  \item $6xy \div \dfrac{2}{3}y = \textcolor{myred}{9x}$
  \item $\dfrac{1}{2}x^2y \div \left(-\dfrac{1}{4}xy^2 \right) = \textcolor{myred}{-\dfrac{2x}{y}}$
\end{spacing}
\begin{spacing}{5}
  \item $a = 5, b = -1$のとき,次の式の値を求めなさい.
        \begin{enumerate}
            \item $3a + 5b = 3\times5+5\times(-1)=\textcolor{myred}{10}$
            \item $5(a - 3b) + 4(-2a + 5b)= -3a+5b=-3\times5+5\times(-1)= \textcolor{myred}{-20}$
            \item $9a^3b \div (-3a^2) = -3ab = -3\times5\times(-1)= \textcolor{myred}{15}$
            \item $8ab^2 \div 4b= 2ab = 2\times5\times(-1)= \textcolor{myred}{-10}$
        \end{enumerate}
    \item $x=-3, y=\dfrac{1}{2}$のとき,次の式の値を求めなさい.
        \begin{enumerate}
            \item $\dfrac{1}{2}(2x-6y) - (x+y)= -4y = -4\times\dfrac{1}{2}=\textcolor{myred}{-2}$
            \item $16x^3y^2 \div (-2x^3y)=-8y = -8\times\dfrac{1}{2}=\textcolor{myred}{-4}$
        \end{enumerate}
  \item $x,y$が次の値の時,$x^2 + 4y$の値を求めなさい.
       \begin{enumerate}
           \item $x = 3, y = 2 \qquad 9+8=\textcolor{myred}{17}$
           \item $x = -4, y = -3 \qquad 16-12=\textcolor{myred}{4}$
       \end{enumerate}
\end{spacing}
  
\end{enumerate}

\vspace{2em}

%\textcolor{myred}{ =}==== 解答欄デザイン(赤枠) =====
%\begin{tcolorbox}[
%  colback=white,
%  colframe=myred,
%  b\textcolor{myred}{ox}rule=0.8p\textcolor{myred}{t},
%  arc=2mm,
%  width=\textwidth
%]
%\vspace{6cm} % 解答スペースの高さ
%\end{tcolorbox}

\end{document}
