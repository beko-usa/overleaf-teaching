\documentclass[a4paper,14pt]{article}  % これは絶対先頭に必要なやつ 用紙サイズ,本文フォント大きさ
\usepackage{setspace}                                %余白をとるためのパッケージ
\usepackage{xeCJK, xcolor, tcolorbox, amsmath, amssymb, geometry}
\geometry{top=18mm,bottom=20mm,left=20mm,right=20mm} %ページマージン


% ===== 色定義(RGB) =====
\definecolor{myred}{RGB}{230,0,18}  %ミャクミャクの赤
\definecolor{myblue}{RGB}{0,104,180}  %ミャクミャクの青

% ===== フォント設定 =====
% 日本語フォント設定(丸ゴ風)
\setCJKmainfont{Noto Sans CJK JP} %日本語の本文
\setCJKsansfont{Noto Sans CJK JP} %sans-serifに対応する日本語フォント
\setCJKmonofont{Noto Sans CJK JP} %等幅フォント(コード等)
%\setCJKmainfont{IPAMincho}  % 標準(明朝)

%英文フォント設定
\setmainfont{Noto Sans}

\sffamily % ← 全体をサンセリフ体(丸ゴ風)に

\setCJKfamilyfont{mincho}{IPAMincho} %フォントにファミリ名を付ける
  %プリアンブルは別ファイルから読み込み(共通設定)

\begin{document}

% ===== タイトルブロック =====
\begin{tcolorbox}[
  colback=myblue!5,     % 背景:薄い青
  colframe=myblue,      % 枠:本青
  boxrule=1pt,
  arc=3mm,
  width=\textwidth
]


\begin{center}                                               %センター揃え
{\Large  \textcolor{myblue}{数学テスト No.3 こたえ}}\\[0.4em]         %テストのタイトル,ナンバー
{ \CJKfamily{mincho} 文字式の利用}\\[0.8em]               %テストの内容 ここだけ明朝
\textcolor{myred}{日付:\underline{\hspace{3cm}} \hfill 点数:\underline{\hspace{2cm}}}
\end{center}
\end{tcolorbox}

\vspace{1em}  %下方にスペース

% ===== 問題部分 =====

% iワーク中2 p22~24より20題抜粋

\begin{enumerate}                                        %自動でナンバーがつく
 \begin{spacing}{3.5}                                      %問題部分の改行幅
  \item 次の等式を[ ]の中の文字について解きなさい.
    \begin{enumerate}  
      \item $m + n = 3 \quad [m] \textcolor{myred}{\qquad m = 3-n}$
      \item $\dfrac{1}{2}xy = 10 \quad [y] \qquad y=\textcolor{myred}{\dfrac{20}{x}}$
      \item $V = \dfrac{1}{3}Sh \quad [h] \qquad \textcolor{myred}{h = \dfrac{3V}{S}}$
      \item $3x + 5y - 8 = 0 \quad [x] \qquad \textcolor{myred}{x=\dfrac{8-5y}{3}}$
      \item $3ab = 6 \quad [b] \qquad \textcolor{myred}{b=\dfrac{2}{a}}$
      \item $c = 4(a+b) \quad [a] \qquad \textcolor{myred}{a=\dfrac{c}{4}-b}$
      \item $\dfrac{2p+q}{5}=r \quad [q] \qquad \textcolor{myred}{q=5r-2p}$      
    \end{enumerate}
  \end{spacing}  
  \vspace{1em}    
  \begin{spacing}{1.8} 
  \item 3つの続いた整数の和は3の倍数になる.このわけを,文字を使って説明する.(\quad)は式を,【\quad】には日本語を入れなさい.
    \begin{tcolorbox}[colback=white, colframe=black, boxrule=0.4pt,                   arc=2mm, left=2mm, right=2mm, top=1mm, bottom=1mm]
    3つの続いた整数のうち,もっとも小さい整数をnとすると,3つの続いた整数は
    $n,(\quad \textcolor{myred}{n+1}\quad),(\quad \textcolor{myred}{n+2} \quad)$ と表される.それらの和は,
      \begin{align*}
          n + (\quad \textcolor{myred}{n+1}\quad) + (\quad \textcolor{myred}{n+2} \quad) &= (\quad \textcolor{myred}{3n+3} \quad)\\
          &=3(\quad \textcolor{myred}{n+1} \quad)
      \end{align*}
    $n+1$は【\quad \textcolor{myred}{整数} \quad】だから,$3(\quad \textcolor{myred}{n+1} \quad)$は3の倍数である.
    したがって,3つの続いた整数の和は3の倍数になる.
    \end{tcolorbox}
 \newpage
  \item 1,3,5のような差が2である3つの整数の和は3の倍数になる.このわけを,文字を使って説明しなさい.
    \begin{tcolorbox}[colback=white, colframe=black, boxrule=0.4pt,                   arc=2mm, left=2mm, right=2mm, top=8mm, bottom=8mm]
    \textcolor{myred}{  
   $n$を整数とすると,差が2である3つの整数は
    $n,\quad n+2 \quad,\quad n+4 \quad$ と表される.それらの和は,
      \begin{align*}
          n + (\quad n+2 \quad) + (\quad n+4 \quad) &= ( \quad 3n+6 \quad)\\
          &=3(\quad n+2 \quad)
      \end{align*}
    $n+2$は整数だから,$3(\quad n+2 \quad)$は3の倍数である.\\
    したがって,差が2である3つの整数の和は,3の倍数になる.}
    \end{tcolorbox}
  
  \vspace{2em}
  
  \item 偶数と奇数の和は奇数になる.このわけを,文字を使って説明しなさい.
    \begin{tcolorbox}[colback=white, colframe=black, boxrule=0.4pt,                   arc=2mm, left=2mm, right=2mm, top=8mm, bottom=8mm]
      \textcolor{myred}{  
   $m,n$を整数とすると,偶数と奇数は
    $2m,\quad 2n+1$  と表される.それらの和は,
          $$2m + 2n+1 = 2( \quad m+n \quad)+1$$
    $m+n$は整数だから,$2(\quad m+n \quad)$は偶数である.
    よって,$2(\quad m+n \quad)+1$は奇数である.\\
    したがって,偶数と奇数の和は,奇数になる.}
    
    \end{tcolorbox}
 \end{spacing} 
\end{enumerate}

\vspace{2em}

\end{document}